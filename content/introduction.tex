%!TeX root = ../ECE8408_Project_Ad_Hoc_Routing.tex
\section{Introduction}\label{sec:introduction}
\IEEEPARstart{M}{obile} Ad-Hoc Networks, or MANETs, are a subset of network communication systems that operate in an Infrastructure-less environment. These networks are well suited to a number of fairly niche use cases, including incident response and disaster recovery, third-world and underdeveloped areas, and for military applications. By virtue of their application, as well as their infrastructure-less nature, these networks require extensive peer-to-peer communications, including peer-based routing. Such a routing algorithm necessarily differs from the conventional Open Shortest Path First (OSPF) and Distance Vector (DV) based algorithms such as are implemented in conventional networks.

\subsection{Routing Algorithm Objectives and Overview}\label{subsec:objectives}

This paper proposes a \textbf{R}outing \textbf{I}mplementation \textbf{C}onsidering \textbf{P}roactive \textbf{R}edundant \textbf{R}outes for \textbf{I}nfrastructureless systems, or \textbf{RIC-PERRI} \cite{perry_rp_2022} for short. The algorithm is intended for critical responders in emergency situations, where reliable connections can quite literally be a life-or-death concern. To this end, the algorithm is based on a proactive routing scheme, wherein routes for all nodes are calculated up front, and a full refresh is performed periodically. This is critical, as a node becoming disconnected could indicate a responder in grave danger, and being unaware of this until some random node needs to send a message to the node in question is unacceptable under these circumstances. To the same end, the algorithm computes both a primary and a secondary route for each node, such that a compromised link does not require a routing data update prior to attempting a retransmission of the message in question. 

In order to ensure reliable transmission of data, the routing algorithm considers Received Signal Strength Indicator (RSSI) values as a factor in computing routes. RSSI is defined in IEEE Standard 802.11 as ``a measure by the PHY of the energy observed at the antenna connector used to receive the current PPDU [Physical Layer Protocol Data Unit]''\cite{lanman_standards_committee_80211-2020_2021}. It should be noted that RSSI is measured by the receiver, as opposed to the majority of routing information, which is provided by the originator of routing data.

Although a reliable connection is crucial to the transmission of data, there is not necessarily any benefit to be gained from one sufficiently reliable connection over another, even if the former is of appreciably higher RSSI than the latter. That is, any link that possess an RSSI over a given threshold is equally suited to the transmission of data. Accordingly, this algorithm will not seek to utilize RSSI as a metric in the route selection process, but rather as a a prerequisite for a link being valid as an option for consideration. The RSSI value threshold for such a decision is parameterized and left to the implementer, but it is suggested that such a value be sufficiently high that it allow for all but guaranteed delivery of data.

Instead, the algorithm shall employ an adapted Distance Vector based approach that is optimized to minimize hop count on all routes. This method was chosen for several reasons. First and foremost, the implementation of a DV-based protocol ensures a lightweight and non resource-intensive routing scheme, especially as compared to a more complex OSPF-based algorithm. Furthermore, MANETs deployed to support critical responders are likely to be fairly compact, meaning that there will likely be low propagation delays relative to the queuing delays within each device. Therefore, minimizing hops will reduce the number of queuing delays encountered, ensuring efficient delivery.

The chief modifications this algorithm proposes to Distance Vector relate to the increased need for constant communication in a critical response environment. The protocol adopts the concept of sequenced routing updates from the Destination-Sequenced Distance Vector (DSDV) protocol, originally proposed in \cite{perkins_highly_1994}. As was previously mentioned, each node also maintains two routes for each destination, ensuring redundancy provided the routes do not share common links. Given that a critical response situation is likely small and concentrated, such a network approximates a full-mesh network, making this likely. Finally, the system implements node failure notifications that are broadcast to all nodes upon detection of an inability to reach a given node, which can be used to trigger application-layer responses such as audible alarms, etc.

\subsection{Organization of this Work}\label{subsec:IntroOrganization}
The remainder of this work is organized into five primary sections. Section \ref{sec:stateMachine} provides a functional specification of the algorithm by means of a Finite State Machine (FSM) and accompanying comments. Section \ref{sec:packetDefinitions} defines the various fields and structures used in the packets sent by this algorithm. Section \ref{sec:informalProof} provides a set of informal proofs to demonstrate the correctness of the algorithm under various conditions and failure modes. Finally, \ref{sec:conclusion} concludes the work with a brief summary of the algorithm, as well as a discussion of avenues for its enhancement in the future.