%!TeX root = ../ECE8408_Project_Ad_Hoc_Routing.tex
\section{Conclusion}\label{sec:conclusion}
The Routing Implementation Considering Proactive Effective Redundant Routes for Infrastructure-less systems, or RIC-PERRI, proposes a routing implementation whereby critical responders can deploy a Mobile Ad-Hoc Network in high-risk situations with added peace of mind stemming from a redundant routing system. By precomputing redundant routes for all nodes in a system, as well as enabling proactive notifications for any detected instances of node failure, the architecture ensures constant connectivity and immediate response to the potentially serious real-life implications of a node failure. The algorithm is derived from the ubiquitous Distance Vector routing protocol, and takes strides where possible to ensure compatibility with the packet formats and conventions utilized by it and many similar protocols to ease implementation and adoption.

\subsection{Avenues for Enhancement}
There exist several avenues whereby the RIC-PERRI algorithm could be enhanced, either in order to increase robustness or to support additional featuresets. The Metric Types field, defined in Field \ref{subsubsec:PDAFMetricValue}, allows for the specification of a metric to be used in the routing algorithm. The algorithm is currently limited to a single metric, but the ability to use multiple metrics could enable additional flexibility in deployment if other criteria, such as number of adjacent links, battery life, etc. were desired to be considered in route selection. Similarly, the Failure Notification could be expanded to enable a node to provide enriched information about why it was disconnecting from the network. Further enhancements could also explore other areas, such as allowing the aggregation of statistical data for periodic analysis by an eccentric computer engineer who needs something to do after they wake up at 4 o'clock on a Saturday morning.